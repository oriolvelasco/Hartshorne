\section{Schemes}

\begin{ex}
	Let $A$ be a ring, let $X = \Spec A$, let $f \in A$ and let $D(f) \subseteq X$ be the open complement of $V((f))$. Show that the locally ringed space $(D(f), \locO_X|_{D(f)})$ is isomorphic to $\Spec A_f$.
\end{ex}

\begin{sol}
	We have to build an isomorphism of locally ringed spaces between $\Spec(A_f)$ and $(D(f), \locO_X|_{D(f)})$. Let $S = \{f^n\}_{n \geq 0}$, so $A_f = S^{-1}A$. Then, we know that there is a bijective correspondence between prime ideals of $A_f$ and prime ideals of $A$ that don't cut $S$, which in this case is equivalent to ideals of $A$ that don't contain $f$ (as $f^n \in \ip \iff f \in \ip$) (Atiyah-MacDonald, Proposition 3.11).

	We have a morphism of rings $\varphi: A \to A_{f}$ that maps $a \mapsto a/1$. It induces an application $f: \Spec (A_f) \to D(f)$, $f(\ip) =  \varphi^{-1}(\ip)$ which is bijective and whose inverse is $f^{-1}: D(f) \to \Spec(A_f)$, and $f^{-1}(\ip) = S^{-1}\ip$. $f$ is continuous (as it was seen at Proposition 2.3). As every ideal in $A_f$ is of the form $S^{-1}\ia$ for a certain $\ia$ ideal of $A$ (also a result in Proposition 3.11 of Atiyah-MacDonald), then the closed sets of $\Spec(A_f)$ are of the form $V(S^{-1}\ia) = \{S^{-1}\ip, \text{ with } \ip \text{ prime ideals of A such that} f \notin \ip, \, \ip \supseteq \ia\}$. Then $(f^{-1})^{-1}(V(S^{-1}\ia)) = V(\ia) \cap D(f)$, which is a closed subset of $D(f)$. In conclusion, $f^{-1}$ is also continuous and so $f$ is an homeomorphism.

	Now we need to build an isomorphism of sheaves $f^{\#}: \locO_X|_{D(f)} \to f_* \locO_{Spec(A)}$. Let $f^{\#}(V): \locO_X|_{D(f)}(V) \to f_* \locO_{Spec(A)}(V)$ map the section $\locO_X|_{D(f)}(V) \ni s : V \to \bigcup_{\ip \in V} A_\ip$ to the section $s': \locO_{\Spec{A_f}}(f^{-1}(V))$ that maps a prime ideal $S^{-1}\ip$ to $s'(S^{-1}\ip) = \frac{a/1}{g/1} \in (S^{-1}A)_{S^{-1}\ip}$ if $s(\ip) = a/g \in A_{\ip}$. To check that this is an isomorphism it's enough to check that the induced application on stalks is an isomorphism (By Exercise 1.2). But we already know by Proposition 2.2 that the stalk at point $\ip$ of $\locO_{\Spec(A)}$ is $A_{\ip}$ and the stalk at point $S^{-1}\ip$ of $\locO_{\Spec(A_f)}$ is $(S^{-1}A)_{S^{-1}\ip}$. 

	Then, it is enough to check that $\phi: A_{\ip} \to (S^{-1}A)_{S^{-1}\ip}$ such that $\phi(\frac{a}{g}) = \frac{a/1}{g/1}$ is an isomorphism of rings.
	\begin{itemize}
		\item \textbf{Surjectivity:} Let $\frac{a/f^n}{g/f^m} \in (S^{-1}A)_{S^{-1}\ip}$. Then if $n \geq m$ we have that 
		\[
			\phi \left (\frac{a}{gf^{n-m}} \right ) = \frac{a/1}{gf^(n-m)/1} = \frac{af^n/f^n}{gf^n/f^m} = \frac{a/f^n}{g/f^m} 
		\]
		because $(af^n/f^n) (g/f^m) - (a/f^n) (gf^n/f^m) = 0$.

		Similarly, if $m \geq n$ and then 
		\[
			\phi \left (\frac{a f^{m-n}}{g} \right ) = \frac{af^{m-n}/1}{g/1} = \frac{af^m/f^n}{gf^m/f^m} = \frac{a/f^n}{g/f^m} 
		\]
		because $(af^m/f^n) (g/f^m) - (a/f^n) (gf^m/f^m) = 0$.

		\item \textbf{Injectivity:} $\frac{a/1}{g/1} = 0 \iff \exists h/f^k \in S^{-1}A$ such that $(h/f^k)(a/1) = 0 \iff ha = 0 \iff \frac{a}{g} = 0$ 
	\end{itemize}

	So we have only left to check that we have indeed a local homeomorphism. Again, it will be enough to check that $\phi^{-1}(V)(S^{-1}\ip (S^{-1}A)_{S^{-1}\ip}) = \ip A_{\ip}$. But this is clear as $\frac{a/1}{g/1} \in S^{-1}(\ip)(S^{-1}(A))_{S^{-1}\ip} \iff a/1 \in S^{-1}\ip \iff a \in \ip \iff a/g \in \ip A_{\ip}$. With all these results we have finally that $(D(f), \locO_X|_{D(f)}) \cong \Spec(A_f)$
\end{sol}

\begin{ex}
	Let $(X, \locO_X)$ be a scheme, and let $U \subseteq X$ be any open subset. Show that $(U, \locO_X|_U)$ is a scheme. We call this the induced scheme structure on the open set $U$, and we refer to $(U, \locO_X|_U)$ as an open subscheme of $X$.
\end{ex}

\begin{sol}
	\textbf{Observation:} Let's observe that if we have an isomorphism of ringed spaces, the restriction to open sets is still an isomorphism. That is, if $(X, \locO_X) \cong (Y, \locO_Y)$ via the isomorphism of locally ringed spaces $(f, f^{\#})$, then given $U$ any open set of $X$, $(U, \locO_X|_U) \cong (U, \locO_Y|_{f(U)})$. It is clear that the restriction of the homeomorphism $f$ to $U$, $f|_U: U \to f(U)$ is still an homeomorphism. In addition, the stalks at $P \in U$ are still the same, and so are the induced morphisms on stalks $f^{\#}_P$, so they're isomorphisms and local and therefore the morphism of sheaves $f^{\#}|_U$ is an is a local isomorphism, and so we have indeed that $(U, \locO_X|_U) \cong (U, \locO_Y|_{f(U)})$.

	Let $P \in U$. As $(X, \locO_X)$ is a scheme, then $\exists V$ an open neighbourhood of $P$ such that $(V,\locO_X|_V) \cong (\Spec (A), \locO_{\Spec(A)})$ for a certain ring $A$. Let $(f, f^{\#})$ denote this isomorphism of locally ringed spaces. As $U \cap V$ is an open set of $V$, then $f(U \cap V)$ is an open set of $\Spec(A)$, so $f(U \cap V) = \bigcup_{i \in I} D(f_i)$, so in particular $f(P) \in D(f_i)$ for a certain $i$. Then, $P \in f^{-1}(D(f)) \subseteq V \cap U \subset U$, so taking into account the observation we have that $(f^{-1}(D(f)), \locO_X|_{f^{-1}(D(f))}) \cong (D(f), \locO_{\Spec(A)}|_{D(f)})$. Finally, using Exercise 2.1 we have the isomorphism $(D(f), \locO_{\Spec(A)}|_{D(f)}) \cong \Spec(A_f)$. So in conclusion for each $P \in U$ there is an open neighbourhood of $P$ which is isomorphic as a locally ringed space to the spectrum of a ring, that is, $(U, \locO_X|_U)$ is a scheme.
\end{sol}

\begin{ex}
	Reduced Schemes. A scheme $(X, \locO_X)$ is reduced if for every open set $U \subseteq X$ the ring $\locO_X(U)$ has no niloptent elements.
	\begin{enumerate}[label=\alph*)]
		\item Show that $(X, \locO_X)$ is reduced if and only if for every $P \in X$ the local ring $\locO_{X,P}$ has no nilpotent elements.
		\item Let $(X, \locO_X)$ be a scheme. Let $(\locO_X)_red$ be the sheaf associated to the presheaf $U \mapsto \locO_X(U)_{red}$ where for any ring $A$, we denote $A_{red}$ the quotient of $A$ by its ideal of nilpotent elements. Show that $(X, (\locO_X)_{red})$ is a scheme. We call it the reduced scheme associated to $X$ and we denote it by $X_{red}$. Show that there is a morphism of schemes $X_{red} \to X$ which is an homeomorphism on the underlying topological spaces. 
		\item Let $f: X \to Y$ be a morphism of schemes, and assume that $X$ is reduced. Show that there is a unique morphism $g: X \to Y_{red}$ such that $f$ is obtained by composing $g$ with the natural map $Y_{red} \to Y$.
	\end{enumerate}
\end{ex}

\begin{sol}
	\begin{enumerate}[label=\alph*)]
		\item \boxed{\Rightarrow} Suppose that $0 \neq \overline{(U,f)} \in \locO_{X,P}$ is a nilpotent element. This implies that $\exists n$ such that $\overline({U, f^n}) = \overline{(U,f)}^n = 0$, that is, $\exists W \subseteq U$ such that $f^n|_{W} = 0$ But restriction morphisms are ring morphisms so $(f|_{W})^n = 0$. Moreover, $f|_W$ is not zero, because that would imply that $\overline{(U,f)} = 0$. So in conclusion $f|_W \in \locO_X(W)$ is nilpotent which is a contradiction.

		\boxed{\Leftarrow} Now suppose that $\locO_X(U)$ has niloptent elements for a certain $U$. That means that $\exists f \neq 0 \in \locO_X(U), \, \exists n \geq 1$ such that $f^n = 0$. Now $\exists P \in U$ such that $\overline{(U,f)} \in \locO_{X,P}$ is not zero, as otherwise we would have an open covering of $U$, $\{W_Q\}_{Q \in U}$ such that $f|_{W_Q} = 0$, which would imply $f = 0$. Then, we have $0 \neq \overline{(U,f)}$, but $0 = \overline{(U, f^n)} = \overline{(U,f)}^n$, so $\locO_{X,P}$ has nilpotent elements, which is a contradiction.

		\item First let's make some observations that will be useful later. 

		\textbf{Obs 1:} Given a morphism of rings $\varphi: A \to B$, it induces a morphism between the reduced rings $\varphi_{red}:A_{red} \to B_{red}$, defined by $\varphi_{red}(\overline{a}) = \overline{\varphi(a)}$. The application is well defined, as the image of a nilpotent element is also nilpotent (let $f \neq 0, n \geq 0$ such that $f^n = 0$; then $(\varphi(f))^n = \varphi(f^n) = 0$). Moreover if $\varphi$ is injective $\varphi_{red}$ is also injective: Indeed, $\varphi_{red}(\overline{f}) = 0 \imp \overline{\varphi(f)} = 0 \imp \exists n$ such that $\varphi(f^n) = (\varphi(f))^n = 0$, and, by injectivity of $\varphi$, $f^n = 0 \imp \overline{f} = 0$. Finally, if $\varphi$ is surjective, $\varphi_{red}$ is also surjective: Given any $\overline{b} \in B_{red}$, $\exists a \in A$ such that $\varphi(a) = b$ so $\varphi_{red}(\overline{a}) = \overline{b}$.

		\textbf{Obs 2:} We will prove that $((\locO_{X})_{red})_P = (\locO_{X,P})_{red}$. Indeed, both sets are pairs of elements $(U,f)$ under a certain equivalence relation. Two pairs $(U,f)$ and $(V,g)$ are equal on $((\locO_{X})_{red})_P$ if and only if $\exists W \subseteq U \cap V$ such that $\overline{f}|_W = \overline{g}|_W \iff \overline{f|_W} = \overline{g|_W} \iff f|_W = g|_W + r$ with $r$ a nilpotent element of $\locO_X(W)$. Replacing $W$ by a smaller neighbourhood if necessary, this happens if and only if the element $(W,r) \in \locO_{X,P}$ is nilpotent, so if and only if the elements $(U,f)$ and $(V,g)$ are related in $(\locO_{X,P})_{red}$.

		\textbf{Obs 3:} Let's prove that $(\Spec(A), (\locO_{\Spec(A)})_{red}) \cong (\Spec(A_{red}), \locO_{\Spec(A_{red})})$. First note that the natural application $A \to A_{red}$ induces an application $f: \Spec(A_{red}) \to \Spec(A)$, which is an homeomorphism (Atiyah-Macdonald, Exercise 1.21 iv.). Now for each open set $U$, $(\locO_{\Spec(A)}(U))_{red}$ is the ring consisting on the applications $s: U \to \bigcup_{\ip \in U} A_{\ip}$ such that $\exists U' \subseteq U$ and $a,f \in A$, $f \notin \iq, \, \, \forall \iq \in U'$ such that $s(\iq) = a/f \in \iq$, for every $\iq \in U'$, module the equivalence relation that two applications are equal if their difference is a nilpotent application (the image of each $\ip$ is nilpotent in $A_{\ip}$). Then $(\locO_{\Spec(A)})_{red}(U)$ is the ring of applications $\phi: U \to \bigcup_{P \in U} ((\locO_{\Spec(A)})_{red})_P$ such that $\forall P \in U$ exists $U' \subseteq U$ neighbourhood of $P$ and $s \in \locO_{\Spec(A)}(U')_{red}$ such that $\forall \iq \in U' \phi(\iq) = s_{\iq}$. Now by Observation 2 we can identify the stalk $((\locO_{\Spec(A)})_{red})_P$ with $(A_{\ip})_{red}$. Therefore, this induces a morphism of sheaves $f^{\#}: (\locO_{\Spec(A)})_{red} \to \locO_{\Spec(A_{red})}$ by the natural local isomorphism $(A_{\ip})_{red} \cong (A_{red})_{\overline{\ip}}$. As these are the stalks and they're isomorphic, then $(f,f^{\#})$ is in fact an isomorphism of locally ringed spaces, so in conclusion 
		\[
			(\Spec(A), (\locO_{\Spec(A)})_{red}) \cong (\Spec(A_{red}), \locO_{\Spec(A_{red})})
		\]

		Now let's proceed to prove that $(X,(\locO_X)_{red})$ is a scheme. Ldt $P \in X$. As $(X, \locO_X)$ is a scheme, then $\exists V$ a neighbourhood of $P$ such that $(V, \locO_X|_{V}) \cong (\Spec(A), \locO_{\Spec(A)})$, so we have an isomorphism of locally ringed spaces $f: \Spec(A) \to V$, $f^{\#}: \locO_X|_V \to f_* \locO_{\Spec(A)}$. As a morphism of sheaves, $f^{\#}$ induces a morphism of rings on every open set: $f^{\#}(W): \locO_X|_V (W) \to f_* \locO_{Spec(A)}(W)$. By Obs 1, we can induce morphisms $f^{\#}(W)_{red}: \locO_X|_V (W)_{red} \to f_* \locO_{\Spec(A)}(W)_{red}$, which are injective by Observation 1. In turn, this induces a morphism on the associated sheaves: $f^{\#}_{red}: (\locO_X|_V)_{red} \to (f_* \locO_{Spec(A)})_{red}$, which is injective by Exercise 1.4a. Let's check that it is also surjective. Indeed, given $\overline{x} \in f_* \locO_{\Spec(A)}(W)$ we know that $\exists \{W_i\}$ open cover of $W$, and $x_i \in \locO_X|_V (W)$ such that $f^{\#}(W_i)(x_i) = x|_{W_i}$. Then, again using observation 1, $f^{\#}_{red}(W_i)(\overline{x_i}) = \overline{x|_{W_i}} = \overline{x}|_{W_i}$, and so $f^{\#}_{red}$ is surjective. Moreover, using observation 2, the induced applications on stalks are in fact the reductions of the applications on stalks, that is $(f^{\#}_{red})_P = (f^{\#}_P)_{red}$ so they're local morphisms because $f^{\#}_P$ are local morphisms and the reduction preserves the correspondence between ideals. Finally let's observe that $(f_* \locO_{\Spec(A)})_{red} = f_*(\locO_{\Spec(A)})_{red}$ and $(\locO_X|_V)_{red} = (\locO_X)_{red}|_{V}$, (it's just a matter of checking that the corresponding rings on any open set are equal). So we have proven that we have a morphism of sheaves $(V,(\locO_X)_{red}|_V) \cong (\Spec(A), (\locO_{\Spec(A)})_{red})$ and using observation 3 and composing morphisms, we have finally 
		\[
			(V,(\locO_X)_{red}|_V) \cong (\Spec(A_{red}), \locO_{\Spec(A_{red})})
		\]

		Indeed, we have proven that every point $P$ has an open neighbourhood that is isomorphic to the spectra of a certain ring, i.e. $(X, (\locO_X)_{red})$ is a scheme.

		Now let's build a mosphim of schemes between $X_{red}$ and $X$. That is, let $f: X \to X$ be the identity application, which is an homeomorphism. Now let $f^{\#}(U): \locO_X(U) \to (\locO_X)_{red}(U)$ be the application that maps $x \in \locO_X(U)$ to the constant application $s: U \to ((\locO_X)_{red})_P$, $s(P) = \overline{x}_P$, where $\overline{x}$ is the image of $x$ in $(\locO_X(U))_{red}$. This induces a morphism of sheaves $f^{\#}: \locO_X \to (\locO_X)_{red}$, with induced applications on stalks $f^{\#}_P: (\locO_X)_P \to ((\locO_X)_{red})_P$. $f^{\#}_P$ maps an element to its reduction modulo the nilpotent element of the stalk (we are using the definition of the morphism $f^{\#}$ and observation 2 to identify $((\locO_X)_{red})_P$ and $(\locO_{X,P})_{red}$), and therefore its a local morphism of rings. Then $(f,f^{\#})$ is a morphism of locally ringed spaces, with $f$ homeomorphism, as desired.

		\item 
	\end{enumerate}
\end{sol}

\begin{ex}
	Let $A$ be a ring and let $(X, \locO_X)$ be a scheme. Given a morphism $F: X \to \Spec(A)$, we have an associated map on sheaves $f^{\#}: \locO_{\Spec(A)} \to f_* \locO_X$. Taking global sections we obtain a homomorphism $A \to \Gamma(X,\locO_X)$. Thus there is a natural map
	\[
		\alpha: \Hom_{\mathfrak{Sch}}(X, \Spec(A)) \to \Hom_{\mathfrak{Rings}}(A,\Gamma(X, \locO_X))
	\]
	Show that $\alpha$ is bijective (cf. (I, 3.5) for an analogous statement about varieties).
\end{ex}

\begin{sol}
	We will define an map $\beta: \Hom_{\mathfrak{Rings}}(A,\Gamma(X, \locO_X)) \to \Hom_{\mathfrak{Sch}}(X, \Spec(A))$. Given $\varphi: A \to \Gamma(X, \locO_X)$ morphism of rings, composing with the maps $\pi_P: \Gamma(X,\locO_X) \to \locO_{X,P}$, $x \mapsto \overline{(X,x)}$ we obtain morphisms of rings $\varphi_P: A \to \locO_{X,P}$ for each $P \in X$.

	As $(X, \locO_X)$ is a scheme, then it is in particular a locally ringed space, so $\locO_{X,P}$ is a local ring for all $P$. Let $\im_P$ be the maximal ideal of $\locO_{X,P}$. Then, we can define an application $f: X \to \Spec(A)$ as $f(P) = \varphi_P^{-1}(\im_P)$. The application is clearly well defined, as the antiimage of a prime ideal is prime. Now we will check that it is a continuous application. It is enough to prove that the antiimage of a basic open subset is open. So let $h \in A, D(h)$ the set of prime ideals of $A$ that don't contain $h$. Then, $f^{-1}(D(h)) = \{P \in X \text{ such that } h \notin \varphi_P^{-1}(\im_P)\}$. But $h \notin \varphi_P^{-1}(\im_P) \iff \varphi_P(h) \notin \im_P \iff \overline{(X,\varphi(h))}$ is a unit in $\locO_{X,P}$ $\iff \exists U_P$ neighbourhood of $P$ and $k \in \Gamma(U_P, \locO_X)$ such that $\varphi(h)|_{U_P} k = 1$. Then, $\varphi(h)_Q \in \locO_{X,Q}$ is a unit $\forall Q \in U_P \imp h \notin \varphi_Q(\im_Q) \, \, \forall Q \in U_P$. That means that $f^{-1}(D(h)) = \bigcup_{P \in f^{-1}(D(h))} U_P$ is an open set, and therefore $f$ is continuous.

	Note that the applications $\varphi_P$ can induce naturally applications $\varphi'_P: A_{\varphi^{-1}_P(\im_P)} \to \locO_{X,P}$ mapping $\frac{a_1}{a_2} \mapsto \frac{\varphi_P(a_1)}{\varphi_P(a_2)}$. These are well defined local ring morphisms. Now let's define a morphism of sheaves $f^{\#}$ between $\locO_{\Spec(A)}$ and $f_*\locO_X$ as follows: Given $s \in \locO_{\Spec(A)}(U)$, $s: U \to \bigcup_{\ip \in U} A_{\ip}$, the composition $\varphi_P' \circ s \circ f: f^{-1}(U) \to \bigcup_{P \in f^{-1}(U)} \locO_{X,P}$ is an element of $\Gamma(f^{-1}(U),\locO_X)$. Moreover, the induced morphisms on stalks are just the applications $\varphi_P'$, which we already know that are local morphisms. In conclusion, we have defined a morphism of schemes 
	\[
		\beta(\varphi) = (f, f^{\#}): (X,\locO_X) \to (\Spec(A), \locO_{\Spec(A)})
	\]

	Now let $(f, f^{\#}):= \beta(\varphi)$. Taking global sections of $f^{\#}$ we obtain a map $\psi: \Gamma(\Spec(A), \locO_{\Spec(A)}) \to \Gamma(X, \locO_X)$. As we saw on Proposition 2.2, $\Gamma(\Spec(A), \locO_{\Spec(A)})$ is isomorphic to $A$ via the map $A \ni a \mapsto s: \Spec(A) \to \bigcup_{\ip \in \Spec{A}} A_{\ip}, s(\ip) = \frac{a}{1}$. So given an element of $A$, $f^{\#}(\Spec(A))(a)$ is the application $s: X \to \bigcup_{P \in X} \locO_{X,P}$ that maps each $P$ to $\overline(X, \varphi(a)) \in \locO_{X,P}$, which can be identified with the element $\varphi(a) \in \Gamma(X, \locO_X)$. This proves that $\alpha(\beta(\varphi)) = \varphi$, so the application $\alpha$ is surjective.

	Reciprocally, let $\varphi:= \alpha((f, f^{\#}))$ a morphism of rings between $A$ and $\Gamma(X, \locO_X)$, that is, $\alpha = f^{\#}(\Spec(A))$. The morphism of schemes $f^{\#}$ induces morphisms on stalks: $f^{\#}_P: (\locO_{\Spec(A)})_{f(p)} \to \locO_{X,P}$ given by $\overline{(U,x)} \mapsto \overline{(f^{-1}(U), f^{\#}(U)(x))}$. Composing with the morphism $i_P: A \mapsto (\locO_{\Spec(A)})_{f(P)}$, $i(a) = \overline{(\Spec(A), a)}$ we obtain a morphism $f^{\#}_P \circ i: A \to \locO_{X,P}$ mapping $a \mapsto \overline{(X, \varphi(a))}$. On the other hand, we have already seen that there are morphisms $A \to \locO_{X,P}$ induced by $\varphi$, that we have named $\varphi_P = \pi_P \circ \varphi$, that are equal to $f^{\#}_P \circ i$. Therefore, we have the following commutative diagram
	\begin{tikzcd}
		A \arrow[r, "i"] \arrow[d, "\varphi"]
		& A_{f(P)} \arrow[d, "f^{\#}_P"] \\
		\Gamma(X, \locO_X) \arrow[r, "\pi_P"]
		& \locO_{X,P}
	\end{tikzcd}

	In particular, the antiimage of the maximal ideal $\im_P$ by the two paths must be equal, so $f(p) = (f^{\#}_P \circ i)^{-1}(\im_P) = (\pi_P \circ \varphi)^{-1} = \varphi_P^{-1}(\im_P)$. This means that the application between topological spaces $\beta(\varphi)$ is the same as the original application $f$. Moreover, it is also a consequence of the commutative diagram that the morphisms induced by $\beta$ on stalks are the same as the originals, that is, $\beta(\varphi)^{\#}_P = f^{\#}_P$. Then, given an open set $U \subseteq \Spec(A)$, $x \in \Gamma(U, \locO_{\Spec(A)}$ we have that $\beta(\varphi)^{\#}(U)(x)_P = f^{\#}(U)(x)_P$ for every $P \in U$. Then, $\beta(\varphi)^{\#}(U)(x)$ and $f^{\#}(U)(x)$ agree when restricted to an open neighbourhood $W_P$ of $P$. These open neighbourhoods $\{W_P\}$ form an open cover of $U$, and by Property 3 of sheaves we have the equality in $U$. In conclusion, we have proven that $\beta(\alpha(f,f^{\#})) = (f,f^{\#})$, so $\alpha$ is injective. This completes the proof, and so we have a bijective correspondence 
	\[
		\alpha: \Hom_{\mathfrak{Sch}}(X, \Spec(A)) \to \Hom_{\mathfrak{Rings}}(A,\Gamma(X, \locO_X))
	\]
\end{sol}

\begin{ex}
	Describe $\Spec(\Z)$ and show that it is a final object for the category of schemes, i.e., each scheme $X$ admits a unique morphism to $\Spec(\Z)$.
\end{ex}

\begin{sol}
	The prime ideals of $Z$ are $\ip = (p)$ with $p$ a prime integer, and $(0)$. All the primes of the first type are maximal ideals, so they're closed points in the ring spectrum. Moreover, every ideal contains the zero ideal, so $(0)$ is a generic point in the ring spectrum, such that its closure is the whole space.

	Given any commutative ring with unity $A$, there exists a unique morphism of rings $\varphi: \Z \to A$, defined by $\varphi(n) = n 1_{A}$. In particular, given a scheme $(X, \locO_X)$, we take $A = \Gamma(X,\locO_X)$, and it exists a unique ring morphism $\Z \to \Gamma(X, \locO_X)$. So, by the bijective correspondence between ring morphisms $B \to \Gamma(X, \locO_X)$ and morphisms of schemes $X \to \Spec(B)$ proven in Exercise 2.4, there exists a unique morphism of schemes $X \to \Spec(\Z)$, so $\Spec(\Z)$ is a final object in the category of schemes.
\end{sol}

\begin{ex}
	Describe the spectrum of the zero ring, and show that it is an initial object for the category of schemes. 
\end{ex}

\begin{sol}
	The zero ring $R = 0$, has only one element, so the only ideal of $R$ is the whole ring and therefore it has no prime ideals. Therefore, $\Spec(R) = \varnothing$, the spectrum of the zero ring is the empty set. 

	Now, given any scheme $(X,\locO_X)$ there is a unique application $\varnothing: \Spec(R) \to X$, which is the empty function, and it is continuous (the antiimage of every set is the empty set which is open). Moreover, the only open set in $\Spec(R)$ is the empty set $U = \varnothing$ so by definition of sheaf $\locO_{\Spec(R)}(U) = 0$ and so for each $U \subseteq X$ we have a unique application $\locO_X(U) \to \locO_{\Spec(R)}(f^{-1}(U)) = 0$ which is the zero application (maps every element to zero). In conclusion, we have proven that there is a unique morphism of schemes $\Spec(R) \to X$, so $\Spec(R)$ is an initial object for the category of schemes.
\end{sol}

\begin{ex}
	Let $X$ be a scheme. For any $x \in X$, let $\locO_X$ be the local ring at $x$ and $\im_x$ its maximal ideal. We define the residual field of $x$ on $X$ to be the field $k(x) = \locO_X/\im_x$. Now let $K$ be any field. Show that to give a morphism of $\Spec K$ to $X$ is it equivalent to give a point $x \in X$ and an inclusion map $k(x) \to K$.
\end{ex}

\begin{sol}
	Let $P \in \Spec K$ be the only point of this topological space. Given a morphism of schemes, $(f,f^{\#}): (\Spec (K), \locO_{\Spec(K)}) \to (X, \locO_X)$, $f$ is completely determined by a point $x \in X$, the image of the only point $P \in \Spec(K)$. In addition, $f^{\#}$ induces a morphism on the stalks $f^{\#}_P: \locO_{X,x} \to K$. It is a local morphism because $(f,f^{\#})$ is a morphism of locally ringed spaces, and so $\ker(f^{\#}_P) = (f^{\#}_P)^{-1}(0) = \im_x$ and therefore we can induce an injective morphism $k(x) = \locO_{X,x}/\im_x \to k$. Reciprocally, given an injection $k(x) \to K$ we can induce a unique local morphism $\locO_{X,x} \to K$ sending an element to its class in $k(x)$ and then applying $\varphi$. We will see now that any morphism $f^{\#}$ on the structure sheafs it totally determined by the induced application on the stalk at $P$. 

	%Reciprocally, given $x \in X$ and $\varphi: k(x) \to k$ injective morphism, this defines a unique morphism of schemes. The application between topological spaces $g: \Spec(K) \to X$ is defined by $g(P) = x$ and it is clearly continuous as $g^{-1}(U) = \varnothing$ if $x \notin U$ or $P$ if $x \in U$, which are both open sets. Moreover, $\varphi$ induces an application $\varphi': \locO_{X,x} \to K$ sending an element to its class in $k(x)$ and the applying $\varphi$. Now let's define a morphism between the structure sheaves. Given an open set $U \subseteq X$, if $x \notin U$ then $g^{-1}(U) = \varnothing$, so the only possible application $\locO_X(U) \to \locO_{\Spec(K)}(f^{-1}(U))$ is the zero application. On the other side, if $x \in U$, we can define an $g^{\#}(U)$ that takes an application $s: U \to \bigcup_{Q \in U}\locO_{X,Q}$ and sends it to $\varphi'(s(x)) \in K = \locO_{\Spec(K)}(f^{-1}(U))$. 

	Indeed, given an open set $U \subseteq X$, if $x \notin U$ then $f^{-1}(U) = \varnothing$, so the only possible application $\locO_X(U) \to \locO_{\Spec(K)}(f^{-1}(U))$ is the zero application. Now let's observe that if $x \in U,V$ and $V \subseteq U$, then the restriction morphisms from $f^{-1}(U)$ to $f^{-1}(V)$ is the identity, and therefore $f^{\#}(U)(s) = f^{\#}(U)(s)|_{f^{-1}(V)} = f^{\#}(V)(s|_V)$. Than means that two elements $s,t \in \locO_X(U)$ have the same image by $f^{\#}(U)$ if and only if they're equal in a neighbourhood of $x$. In conclusion, the image of an element $s$ by $f^{\#}(U)$ is equal to the image of its stalk at $x$ by the application $f^{\#}_P$. 

	In conclusion, a morphism of schemes $(f,f^{\#}): (\Spec (K), \locO_{\Spec(K)}) \to (X, \locO_X)$ is completely determined by a point $x \in X$ and an injection $k(x) \to K$.
\end{sol}

\begin{ex}
	Let $X$ be a scheme. For any point $x \in X$ we define the Zariski tangent space $T_X$ to $X$ at $x$ to be the dual space of the $k(x)$-vector space $\im_x/\im_x^2$. Now assume that $X$ is a scheme over a field $k$, and let $k[\epsilon]/\epsilon^2$ be the ring of dual numbers over $k$. Show that to give a $k$-morphism of $\Spec k[\epsilon]/\epsilon^2$ to $X$ is equivalent to giving a point $x \in X$ rational over k (i.e., such that $k(x) = k$), and an element of $T_x$.
\end{ex}

\begin{ex}
	If $X$ is a topological space, and $Z$ an irreducible closed subset of $X$, a generic point for $Z$ is a point $\ $ such that $Z = \overline{\{\zeta\}}$. If $X$ is a scheme, show that every (nonempty) closed subset has a unique generic point.
\end{ex}

\begin{sol}
	Foreach $P \in Z$, let's consider $V_P$ the open neighbourhood of $P$ such that $(V_P, \locO_X|_{V_P})$ is isomorphic to the spectrum of a given ring. Let's fix $P \in Z$ and let $A$ such that $(V_P, \locO_X|_{V_P}) \cong (\Spec(A), \locO_{\Spec(A)})$. Let $f$ be the homeomorphism $f: V_P \to \Spec(A)$.

	Let's observe that $Z \cap V_P$ is irreducible (as an open set of $Z \cap V_P$ is of the form $U \cap Z \cap V_P$ and then $(U_1 \cap Z \cap V_P) \cap (U_2 \cap Z \cap V_P) = (Z \cap V_P) \cap (Z \cap U_1) \cap (Z \cap U_2)$ which is non empty because it's the intersection of nonempty open sets of $Z$, which is irreducible.

	As $f$ is an homeomorphism, then $f(Z\cap V_P)$ is irreducible and closed (as a subset of $\Spec(A)$). As it is closed, $\exists \ia$ ideal of $A$ such that $f(Z \cap V_P) = V(\ia) = V(r(\ia))$. Now let $fg \in r(\ia) \imp fg \in \iq \, \, \forall \iq \in V(\ia) \imp D(fg) \cap V(\ia) = \varnothing$. As $D(fg) = D(f) \cap D(g)$ (Atiyah-MacDonald, exercise 1.17), then $(D(f) \cap V(\ia)) \cap (D(g) \cap V(\ia)) = \varnothing$. By irreducibility of $V(\ia)$, either $D(f) \cap V(\ia)$ or $D(g) \cap V(\ia)$ must be empty, so either $f$ or $g$ belong to $\iq, \, \, \forall \iq \in V(\ia)$. In conclusion, $r(\ia)$ is a prime ideal, that we will name $\ip$, and $f(Z \cap V_P) = V(\ip) = \overline{\{\ip\}}$, and $\ip$ is the only point of $\Spec(A)$ with this property. Then, $f^{-1}(\ip) = Q_P \in V_P \cap Z$, and as the closure of image is the image of the closure under an homeomorphism, then $\overline{\{Q_P\}} = Z \cap V_P$, where the closure here is the closure in $V_P$. Then, the closure of $Q_P$ in $Z$ is $\overline{\{Q_P\}} = Z \cap \overline{V_P}$.

	Now note that $Z \setminus (Z \cap \overline{V_P})$ and $Z \cap V_P$ are open sets of $Z$, and their intersection is empty. As $Z$ is irreducible, one of them must be empty. $Z \cap V_P$ is not empty, as $P$ belongs to this subset, so we must have $Z \setminus (Z \cap \overline{V_P}) = \varnothing \imp Z \cap \overline{V_P} = Z$. So, in conclusion, $\overline{\{Q_P\}} = Z$ and we have proved the existence of a generic point of $Z$. Now let's prove the uniqueness. Note that the point $Q_P$ is unique with this property in $V_P \cap Z$ (as $\ip$ was unique, as we already observed), but we could have a different point $Q_P$ for each open set $V_P$. However, as $\overline{\{Q_P\}} = Z, \, \, \forall P' \in Z$ and $\forall U$ open neighbourhood of $P'$, $Q \in U$. In particular, taking $U = V_{P'}$ we have that $Q_P \in V_{P'}$, and therefore $Q_P = Q_{P'}$ (via the composition of homeomorphisms from $V_P$ and $V_{P'}$ to the corresponding ring spectrums) and so the generic point is unique, $Q_P = \zeta \, \, \forall P$.
\end{sol}

\begin{ex}
	Describe $\Spec(\R[x])$. How does it compare to the set $\R$? To $\C?$
\end{ex}

\begin{sol}
	$\R$ is a field, and so $\R[x]$ is a principal ideal domain, so $\Spec(\R[x]) = (0) \cup \mathrm{Max}(\R[x])$. Then, the prime ideals of $\R[x]$ are the zero ideal and ideals of the form $(f)$, with $f$ an irreducible polynomial (which are all maximal ideals). Then $\Spec(\R[x])$ has a generic point $(0)$ and a closed point for each irreducible polynomial. We know that irreducible polynomials in $\R[x]$ can be of two types: $f = x-a$, with $a \in \R$, or $f = (x-(a+bi))(x-(a-bi))$, $a,b \in \R, b \neq 0$. Therefore, in $\Spec(\R[x])$ we have a closed point for each $a \in \R$ and a closed point for each pair of complex conjugates. Then, $\Spec(\R[x])$ can be identified with the closed upper half plane plus a generic point $(0)$.
\end{sol}

\begin{ex}
	Let $k = \mathbb{F}_p$ be the finite field with $p$ elements. Describe $\Spec k[x]$. What are the residue fields of its points? How many points are there with a given residue field?
\end{ex}

\begin{sol}
	As $k$ is a field, following the same reasoning of last exercise, the spectrum of $k[x]$ will consist of a closed point for each irreducible polynomial in $k[x]$ plus a generic point corresponding to the ideal $(0)$. There are irreducible polynomials in $k[x]$ of arbitrary degree $n$: Indeed, we can consider an extension of degree n of $\mathbb{F}_p$, $\mathbb{F}_{p^n}$. As the multiplicative group of $\mathbb{F}_{p^n}$ is finite, it is cyclic, so $\mathbb{F}_{p^n} = \{0, \alpha, \dots, \alpha^{p^{n-1}(p-1)} \}$, and then $\mathbb{F}_{p^n} = \mathbb{F}_p(\alpha)$. As $[\mathbb{F}_{p^n} : \mathbb{F}_p] = n$, then $\text{Irr}(\alpha, \mathbb{F}_p, x)$ has degree $n$.

	Then, given $f(x) \in k[x]$ irreducible of degree $n$, the residue field at the point corresponding to $(f)$ is $k[x]_{(f)} / (f)k[x]_{(f)}$. As localization and quotient commute, this field is the same as $(k[x]/(f))_{(f)}$, but $k[x]/(f)$ is already a field, $\mathbb{F}_{p^n}$ so every element is already invertible, and when we localize we are not adding any elements. In conclusion, the residue field at the point corresponding to the ideal $(f)$ is $\mathbb{F}_{p^n}$, where $n$ is the degree of the polynomial. The residue field at the generic point $(0)$ is the field of fractions of the ring $\mathbb{F}_p$, that is, $\mathbb{F}_p(x)$. 

	Then, the number of points with a given residue field is exacly the same as the number of irreducible polynomials of $k[x]$ with a given degree $n$. This number is given by the expression $\frac{1}{n} \sum_{d \mid n} \mu(n/d) p^d$, where $\mu$ is the Möbius function: $\mu(n) = 0$ if $p^2 | n$ for a certain prime $p$, and $\mu(n) = (-1)^k$ otherwise, where $k$ is the number of different prime factors of $n$.
\end{sol}

\begin{ex}
	Glueing Lemma. Generalize the glueing procedure described in the text (2.3.5) as follows. Let $\{X_i\}$ be a family of schemes (possible infinite). For each $i \neq j$, suppose given an open subset $U_{ij} \subseteq X_i$, and let it have the induced scheme structure (Ex. 2.2). Suppose also given for each $i \neq j$ an isomorphism of schemes $\varphi_{ij}: U_{ij} \to U_{ji}$ such that (1) for each $i,j$ $\varphi_{ij} = \varphi_{ji}^{-1}$ and $(2)$ for each $i,j,k$, $\varphi_{ij}(U_{ij} \cap U_{ik}) = U_{ji} \cap U_{jk}$ and $\varphi_{ik} = \varphi_{jk} \circ \varphi_{ij}$ on $U_{ij} \cap U_{ik}$. Then show that there is a scheme $X$, together with morphisms $\psi_i: X_i \to X$ for each i, such that (1) $\psi_i$ is an isomorphism of $X_i$ onto an open subscheme of $X$, (2) the $\psi_i(X_i)$ cover $X$, (3) $\psi_i(U_{ij}) = \psi_i(X_i) \cap \psi_j(X_j)$ and (4) $\psi_i = \psi_j \circ \varphi_{ij}$ on $U_{ij}$. We say that $X$ is obtained by glueing the schemes $X_i$ along the isomorphisms $\varphi_{ij}$. An interesting special case is when the family $X_i$ is arbitrary, but the $U_{ij}$ and $\varphi_{ij}$ are all empty. Then the scheme $X$ is called the disjoint union of the $X_i$ and its denoted $\bigsqcup X_i$.
\end{ex}

\begin{sol}
	We will proceed as in 2.3.5. Let's consider $X$ the topological space $\bigsqcup X_i$ modulo the equivalence relation $\varphi_{ij}(x) ~ x$, $\forall x \in U_{ij}, \, \, \forall i,j$. We have maps $\psi_i$ which send each element $x \in X_i$ to its equivalence class in $X$. Then, we define a topology on $X$ as follows: A set $V \subseteq X$ is open if and only if $\psi_i^{-1}(V)$ is open for each $i$. Its clear that this topology makes the applications $\psi_i$ continuous.

	%The finest topology that makes the canonical projection map $\pi: \bigsqcup X_i \to X$, $x \mapsto \overline{x}$ continuous. Its clear that the inclusion maps $X_i \to \bigsqcup X_i$ are continuous, and therefore  (the composition of the inclusion and $\pi$), and they're continuous for being the composition of continuous applications. 

	Again following the developement of 2.3.5 we can define a sheaf of rings on $X$ as 
	\[
		\locO_X(V) = \{(s_i) \in \prod_i \locO_{X_i}(\psi_i^{-1}(V)) \text{ such that } \varphi_{ij}(s_i|_{\psi_i^{-1}(V) \cap U_{ij}}) = s_j|_{\psi_j^{-1}(V) \cap U_{ji}} \}
	\]
 
	Now, to avoid checking step by step that it is indeed a sheaf of rings, we will make use of Exercise 1.22. Note that $\psi_i(X_i)$ are open sets of $X$, as $\psi_j(\psi_i(X_i)) = X_i$ if $i = j$, and $U_{ji}$ if $j \neq i$. Then, we can define on $\psi_i(X_i)$ a sheaf by direct image, $(\psi_i)_* \locO_{X_i}$. Moreover, $\psi_i(X_i) \cap \psi_j(X_j)$ are the elements of $X$ that admit representatives both in $X_i$ and $X_j$, so $\psi_i(X_i) \cap \psi_j(X_j) = \psi_i(U_{ij}) = \psi_j(U_{ji})$. Therefore, we have isomorphisms of sheaves 
	\[
		\phi_{ij} := {\varphi_{ij}^{\#}}_* : (\psi_j)_* \locO_{X_j}|_{\psi_i(X_i) \cap \psi_j(X_j)} \to (\psi_i)_* \locO_{X_j}|_{\psi_i(X_i) \cap \psi_j(X_j)}
	\]

	And those morphisms satisfy $\phi_{ik} = \phi_{jk} \circ \phi_{ij}$ on $\psi_i(X_i) \cap \psi_j(X_j) \cap \psi_k(X_k)$. Then, we are under the situation of Exercise 1.22, and the sheaf we have defined on $X$ is the same as the sheaf defined by glueing the sheafs $(\psi_i)_* \locO_{X_i}$ defined in the open cover $\psi_i(X_i)$. So we already know that $\locO_X$ is a sheaf. 

	In addition, let's observe that $(\psi_i)_* \locO_{X_i}$ and $\locO_{X_i}$ are isomorphic sheaves, as $\psi_i$ is injective (the equivalence relation that defines $X$ only identifies points from different $X_i$, so each element of $X$ has at most one representative in $X_i$), so $\psi_i$ is an homeomorphism when we restrict the image to $\psi(X_i)$ instaed of $X$.

	Moreover, it is clear that for each point $P \in X$, $P = \psi_i(Q)$ for a certain $i$ and $Q \in X_i$. Then, the ring $\locO_{X,P}$ is isomorphic to the local ring $\locO_{X_i, Q}$ and so it is local. Moreover, every point has an open set isomorphic to the spectrum of a ring (as the image of an open set of $X_i$ by $\psi_i$ is an open set of $X$). This proves that $(X, \locO_X)$ is a scheme.

	\vspace{3mm}

	Now let's consider the morphisms of shemes $(\psi_i, \psi_i^{\#}): (X, \locO_X) \to (X_i, \locO_{X_i})$, where $\psi_i^{\#}$ is the projection on the $i-th$ coordinate. We have already commented that $(X_i, \locO_{X_i})$ and $(\psi(X_i), (\psi_i)_*\locO_{X_i})$ are isomorphic schemes. We also know that $\locO_{X}|_{\psi_i(X_i)} \to (\psi_i)_*\locO_{X_i}$ is an isomorphism of sheaves by Exercise 1.22. Then the composition, which is $\psi_i^{\#}|_{\psi(X_i)}$ is an isomorphism of sheaves, and so we have that $(\psi_i, \psi_i^{\#})$ is an isomorphism of schemes from $X_i$ to an open subscheme of $X$, corresponding to $\psi_i(X_i)$. This proves property (1). We have already proven properties (2) and (3) earlier. With respect to (4), it follows from the fact that $\phi_{ik} = \phi_{jk} \circ \phi_{ij}$, which is proved in 1.22, toghether with the definition of the sheaf morphisms $\phi_{ij}$. 
\end{sol}

\begin{ex}
	A topological space is quasi-compact if every open cover has a finite subcover. 
	\begin{enumerate}[label=\alph*)]
		\item Show that a topological space is noetherian if and only if every open subset is quasi-compact.
		\item If $X$ is an affine scheme, show that $\mathrm{sp}(X)$ is quasi-compact, but not in general noetherian. We say that a scheme $X$ is quasi-compact if $\mathrm{sp}(X)$ is.
		\item If $A$ is a noetherian ring, show that $\mathrm{sp}(\Spec(A))$ is a noetherian topological space.
		\item Give an example to show that $\mathrm{sp}(\Spec(A))$ can be noetherian even when $A$ is not.
	\end{enumerate}
\end{ex}

\begin{sol}
	\begin{enumerate}[label=\alph*)]
		\item $\boxed{\Rightarrow}$ Let $X$ be a noetherian topological space. Every open subset $U$ of $X$ is noetherian, as open sets of $U$ are also open sets of $X$, so every ascending chain of open sets of $U$ must stabilize, and so $U$ is indeed noetherian. Then, it is enough to prove that a noetherian topological space is quasi-compact. Now let $\{U_{\alpha}\}$ be an open cover of $X$. Now let's build an ascending open chain: We pick an open set $U_1 \in \{U_{\alpha}\}$. If $U_1 \neq X$, there mus exist another open set $U_2 \in \{U_{\alpha}\}$ such that $U_2 \not\subseteq U_1$. Then $U_1 \subset U_1 \cup U_2$. Again, if $U_1 \cup U_2 \neq X$, $\exists U_3 \not\subseteq U_1 \cap U_2$. Repeating this steps inductively we build an ascending chain
		\[
			U_1 \subset U_1 \cup U_2 \subset \dots \subset \bigcup_{i = 1}^n U_i \subset \dots
		\]

		As $X$ is noetherian, the chain must stabilize at some point $m$ and so we must have $\bigcup_{i = 1}^m U_i = X$, so $X$ is quasi-compact.

		$\boxed{\Leftarrow}$ Consider a chain of open sets of $X$, $U_1 \subseteq U_2 \subseteq \dots \subseteq U_n \subseteq \dots$. Then $U = \bigcup_{i = 1}^\infty U_i$ is an open set, and $\{U_i\}_{i = 1}^{\infty}$ is an open cover of $U$. $U$ is quasi-compact as it is an open subset, so $\exists U_{i_1}, \dots U_{i_m}$ that cover $U$. Therefore, the chain stabilizes at point $i_m$, and therefore $X$ is noetherian.

		\item If $X$ is an affine scheme, then $X = \Spec(A)$ for a certain ring $A$. As the sets $D(f)$ form a base of the topology, every open covering of $X$ can be regarded as a basic open covering. So let $X = \bigcup_{i \in I} D(f_i) = \bigcup_{i \in I} X \setminus V(f_i) = X \setminus \bigcap_{i \in I} V(f_i)$. This happens if and only if $V(\bigcup_{i \in I}f_i = \bigcap_{i \in I} V(f_i) = \varnothing \iff$ the ideal generated by the elements $f_i$ is the whole ring $A$. That means that $\exists J$ a finite subset of $I$ such that $1 = \sum{i \in J}g_i f_i$. Then, following the implications in the opposite direction, we have $\bigcap_{i \in J} V(f_i) = \varnothing \imp X \setminus \bigcap_{i \in J}V(f_i) = \bigcup_{i \in J} X \setminus V(f_i) = \bigcup_{i \in J} D(f_i) = X$. So the set $\{D(f_i)\}_{i \in J}$ is a finite open subcovering of $\{D(f_i)\}_{i \in I}$. That shows that $\mathrm{sp}(X)$ is quasi-compact.

		Now consider the ring $A = k[x_1, \dots, x_n \dots]$, that is a ring of polynomials with infinite indeterminates over a field $k$. The ideals $(x_1) \subset (x_1, x_2) \subset \dots \subset (x_1, x_2, \dots, x_n) \subset \dots$ are all prime and form an ascending chain that doesn't stabilize. Then, $V(x_1) \supset V(x_1, x_2) \supset \dots \supset V(x_1, x_2, \dots, x_n) \supset \dots$ is a descending chain of closed sets of $\text{sp}(X)$ which doesn't stabilize, and so the topological space is not noetherian.

		\item Let $V(\ia_1) \supseteq V(\ia_2) \supseteq \dots$ be a descending chain of closed sets in $\mathrm{sp}(\Spec(A))$. That means that we have a corresponding chain on ideals $r(\ia_1) \subseteq r(\ia_2) \subseteq \dots$. As the ring is noetherian, this chain must stabilize. Then, taking into account that $V(\ia) = V(r(\ia))$, the initial descending chain of closed sets also stabilizes, so the topological space $\Spec(A)$ $\mathrm{sp}(\Spec(A))$ is noetherian.

		\item Let's consider the ring $A = k[x_1, \dots, x_n \dots] / (x_1, x_2^2, \dots, x_n^n \dots)$. These ring has only one prime ideal, $(x_1, x_2, \dots, x_n, \dots)$, so $\mathrm{sp}(\Spec(A))$ has only one point, and therefore every descending chain of closed sets must stabilize and $\mathrm{sp}(\Spec(A))$ is noetherian. However, the chain of ideals of $A$ $(x_1) \subset (x_1, x_2) \subset \dots$ doesn't stabilize, and so $A$ is not a noetherian ring.
	\end{enumerate}

\end{sol}

\begin{ex}
	\begin{enumerate}[label=\alph*)]
		\item Let $S$ be a graded ring. Show that $\Proj(S) = \varnothing$ if and only if every element of $S_{+}$ is nilpotent.
		\item Let $\varphi: S \to T$ be a morphism of graded rings (preserving degrees). Let $U = \{\ip \in \Proj(T) | \ip \not\supseteq \varphi(S_+)\}$. Show that $U$ is an open subset of $\Proj(T)$, and show that $\varphi$ determines a natural morphism $f: U \to \Proj(S)$.
		\item The morphism $f$ can be an isomorphism even when $\varphi$ is not. For example, suppose that $\varphi_d: S_d \to T_d$ is an isomorphism for all $d \geq d_0$, where $d_0$ is an integer. Then show that $U = \Proj(T)$ and the morphism $f: \Proj(T) \to \Proj(S)$ is an isomorphism.
		\item Let $V$ be a projective variety with homogeneous coordinate ring $S$. Show that $t(V) \cong \Proj(S)$. 
	\end{enumerate}
\end{ex}

\begin{sol}
	\begin{enumerate}[label=\alph*)]
		\item Suppose that every element of $S_+$ is nilpotent, and let $\ip$ be an homogeneous prime ideal. Given $x \in S_+$, $x^n = 0 \in \ip$ for a certain integer $n$, so we must have $x \in \ip$. This means that $\ip \supseteq S_+$ and therefore $\Proj(S) = \varnothing$, as it is the set of homogeneous prime ideals that don't contain $S_+$. Reciprocally, suppose that $\Proj(S) = \varnothing$. This means that every homogeneous prime ideal of $S$ contains $S_+$. 

		Reciprocally, let now $\Proj(S) = \varnothing$. Let's first prove that every prime ideal $\ip$ contains an homogeneous prime ideal $\ip_h$, the one generated by the homogeneous elements of $\ip$. It is clear that $\ip_h$ is homogeneous and that $\ip_h \subseteq \ip$. We need to prove that $\ip_h$ is prime. Let $f,g$ such that $fg \in \ip_h$. As $S$ is a graded ring, then $f$ and $g$ admit a unique descomposition as a sum of homogeneous elements, $f = \sum_{i = 0}^n f_i$ and $g = \sum_{i = 0}^m g_i$. We have to prove that either $f$ or $g$ belong to $\ip_h$. We will proceed by induction on $n+m$. If $n = m = 0$, then, $fg \in \ip_h \subseteq \ip \imp$ $f$ or $g$ $\in \ip$, and as they're homogeneous elements, they belong to $\ip_h$. In the general case, $fg = f_n g_m + (\sum_{i = 0}^{n-1} f_i)(\sum_{i = 0}^{m-1} m_i) + f_n(\sum_{i = 0}^{m-1} g_i) + g_m(\sum_{i = 0}^{n-1} f_i)$. As the ideal $\ip_h$ is homogeneous, then the component of degree $i$ of $fg$ belongs to $\ip_h$ for each $i$. In particular, the component of degree $nm$, which is $f_n g_m$ belongs to $\ip_h \subseteq \ip$, so either $f_n$ or $g_m$ belong to $\ip$, and therefore $\in \ip_h$ as they're homogeneous. Suppose that $f_n \in \ip_h$. Then $\ip_h \ni fg - f_ng = (\sum_{i = 0}^{n-1} f_i)(\sum_{i = 0}^m g_i)$. Applying induction hypothesis, either $g$ or $f-f_n$ belong to $\ip_h$. As we already know that $f_n \in \ip_h$, then we have indeed that $f$ or $g$ belong to $\ip_h$, so it is a prime ideal.

		The fact that every prime ideal contains an homogeneous prime ideal implies that the nilradical of $S$ van be calculated as the intersection of all homogeneous prime ideals. As $\Proj(S) = \varnothing$, every homogeneous prime ideal contains $S_+$, and therefore the nilradical contains $S_+$, so every element of $S_+$ is nilpotent.

		\item If $\varphi$ is a graded morphism, the image of an homogeneous element is homgeneous, and therefore, if $\ia$ is an homogeneous ideal of $S$, $\varphi(\ip)^e$ is an homogeneous ideal of $T$. Note also that $\ip \supseteq \varphi(\ia) \iff \ip \supseteq \varphi(\ia)$. Then, $V(\varphi(S_+)^e)$ is a closed subset of $\Proj(T)$ and therefore its complementary, which is $U$, is open.

		It is clear that $\varphi$ defines an application between prime ideals of $T$ and prime ideals of $S$, that maps $\ip \mapsto \varphi^{-1}(\ip)$. The contraction of an homogeneous ideal $\ip$ is homogeneous: Indeed, let $f = f_0 + \dots + f_n$ such that $\varphi(f) = x$, with $x$ homogeneous of degree $d$, then $\varphi(f_i) = 0$ if $i \neq d$ and $\varphi(f_i) = x$ if $i = d$. So $f_i, \, i \neq d \in \varphi^{-1}(\ip)$ as they belong to the kernel of $\varphi$ and $f_d \in \varphi^{-1}(x)$. So the components of each element of $\varphi^{-1}(\ip)$ belong to the ideal, and therefore $\varphi^{-1}(\ip)$ is homogeneous. This means that we can restrict the application between prime ideals of $T$ and prime ideals of $S$ to homogeneous prime ideals. In addition, $\ip \not\supseteq \varphi(S_+) \imp \imp \varphi^{-1}(\ip) \not\supseteq \varphi^{-1}(\varphi(S_+))$. We know that $\varphi^{-1}(\varphi(S_+)) \supseteq S_+$. But $\varphi(S_+)$ doesn't have elements of degree $0$, so its antiimage doesn't have elements of degree $0$, which means that $\varphi^{-1}(\varphi(S_+)) = S_+$. In conclusion, $\ip \not\supseteq \varphi(S_+) \imp \varphi^{-1}(\ip) \not\supseteq S_+$. Then, the images of elements of $U$ belong to $\Proj(S)$, so when we restrict the application $\varphi^{-1}$ to $U$ we obtain an application $f: U \to \Proj(S)$.

		Now let's check that this is a continuous map between the corresponding topological spaces (with $U$ having the induced topology by $\Proj(T)$). It is enough to check that the antiimage of a basic open set is open. Let $D_+(h)$ be a basic open subset of $\Proj(S)$, (with $h \in S$). Then, $f^{-1}(D_+(h)) = \{\ip \in U \text{ such that } h \notin \varphi^{-1}(\ip) \}$ As $h \notin \varphi^{-1}(\ip) \iff \varphi(h) \notin \ip$, then $f^{-1}(D_+(h))$ is equal to $D_+(\varphi(h)) \cap U$, which is open in $U$. Then the application $f$ is indeed continuous.

		Let's define an associated morphism of sheaves as follows: $f^{\#}: \locO_{\Proj(S)} \to f_*\locO_{\Proj(T)}|_U$. Given an application $s: V \to \bigsqcup_{\ip \in V} S_{(\ip)} \in \locO_{\Proj(S)}$ we map it to the application $f^{\#}(s) = \varphi_{\ip} \circ s \circ f: f^{-1}(V) \to \bigsqcup_{\ip \in f^{-1}(V)} T_{(\ip)}$, where $\varphi_{\ip}: S_{\varphi^{-1}(\ip)} \to T_{\ip}$ is the localization of $\varphi$ (as $\varphi$ preserves degrees, $\varphi_{\ip}$ also does, and therefore takes elements of degree zero to elements of degree zero, and so $f^{\#}$ is well defined). $f^{\#}$ is in addition a morphism of locally ringed spaces, as the induced applictions on stalks are the local ring morphisms $\varphi_{\ip}$. In conclusion, $(f, f^{\#})$ is a morphism of schemes between $U$ and $\Proj(S)$.

		\item If $\ip \supseteq \varphi(S_+)$ and $\varphi_d$ is an isomorphism, then $\ip \supseteq T_d$, $\forall d \geq d_0$. Now, given $x \in T_+$, $\exists n$ such that $x^n \in \ip$ ($n$ such that $n \mathrm{deg}(x) \geq d_0$). As $\ip$ is prime, $x^n \in \ip \imp x \in \ip$. So $\ip \supseteq \varphi(S_+) \iff \ip \supseteq T_+$, and therefore $U = \Proj(T)$.

		Now let's make 2 observations that will be useful to prove that $f$ is a morphism. 
		\textbf{Obs 1:} Every $\ip$ homogeneous prime ideal of $S$ that doesn't contain $S_+$ cannot contain all $T_d$ for arbitrary degree $d \geq d_0$, as therefore, $\forall x \in T_d, d > 0, \exists n$ such that the degree of $x^n$ is $\geq d_0$, and thereore $x^n \in \ip \imp x \in \ip$, and all $S_+$ is contained in $\ip$, which is a contradiction.

		\textbf{Obs 2:} As $\varphi_d$ is an isomorphism, $\varphi(a^n) \in \varphi(\ip) \cap T_d$, with $d \geq d_0$ implies that $a^n \in \ip$, as $\varphi$ is injective. Then $a^n \in \ip \imp a \in \ip$. 

		\begin{itemize}
			\item $f$ is injective: Let $\ip, \iq$ homogeneous prime ideals of $T$ such that $f(\ip) = f(\iq) \iff \varphi^{-1}(\ip) = \varphi^{-1}(\iq)$. Let $x \in \ip$ homogeneous of degree $> 0$. Then, $\exists n$ such that $d:= \mathrm{deg}(x^n) \geq d_0$. Then, as $\varphi_d$ is an isomorphism, $\exists! a \in \varphi_d^{-1}(x^n) \subseteq \varphi_d^{-1}(\ip \cap T_d) = \varphi^{-1}(\iq \cap T_d)$. Therefore, $\varphi_d(a) = x^n \in \iq \imp x \in \iq$ as $\iq$ is prime. Now let $x \in \ip$ of degree $d = 0$. Now let's take y of homogeneous of degree $\geq d_0$, $y \notin \iq$, which exists by Observation 1. Then, $yx \in \ip$ has degree $\geq d_0$ and therefore $\exists! a \in S_d$ such that $a \in \varphi^{-1}(xy) \in \varphi^{-1}(\ip) = \varphi^{-1}(\iq)$. That implies that $xy \in \iq$, but as $y \notin \iq$ and $\iq$ is prime, then $x \in \iq$. This proves $\ip \subseteq \iq$ and by the symmetric argument, the two ideals are equal so $f$ is injective.

			\item $f$ is surjective: Let $\ip$ be a prime ideal of $\Proj S$. Let $\iq$ be the ideal generated by the homogeneous elements of $T$ such that $x^n \in \varphi(\ip)$ for a certain $n$, or homogeneous elements of degree zero of $T$ such that $xy \in \varphi(\ip)$ for a certain $y$ such that $y \notin \varphi(\ip)$. The ideal $\iq$ is homogeneous (because it's generated by homogeneous elements). Let's check that it is a prime ideal: Given $x,y$ homogeneous elements of degree $\geq 1$, such that $xy \in \iq \imp \exists n$ such that $(xy)^n \in \varphi(\ip)$. We can assume that $n \geq d_0$ as $a^n \in \varphi(\ip) \imp a^{nm} \in \varphi(\ip) \, \, \forall m \geq 1$. Then, as $x^n$ and $y^n$ are elements of $T_d$, $d \geq 0$, and $\varphi_d$ is surjective $\imp \exists! a,b \in S$ such that $\varphi(a) = x^n, \, \varphi(b) = y^n$. Then $\varphi(ab) \in \varphi(\ip)$ and as $\varphi_d$ is injective for $d \geq d_0$, then $ab \in \ip$. As $\ip$ is prime either $a$ or $b \in \ip \imp x^n$ or $y^n$ belong to $\varphi(\ip)$ and therefore $x$ or $y$ belong to $\iq$. Now let $x$ have degree zero, and $xy \in \iq$. If $y$ has degree $0$, then this means that $\exists z$ such that $xyz \in \varphi(\ip)$. Then $yz$ has degree $d$ and $x(yz) \in \varphi(\ip)$, which means that $x \in \iq$. If $y$ has degree different from $0$, then $(xy)^n \in \varphi(\ip)$. If $y^n \notin \varphi(\ip)$ then $x \in \iq$. If $y^n \in \varphi(\ip)$, then $y \in \iq$. In conclusion, $\iq$ is a prime ideal.

			Now let $x \in \iq$. If $\mathrm{deg}(x) \geq 1$ then $\exists n$ such that $x^n \in \varphi(\ip) \imp x^n = \varphi(a)$, with $a \in \ip$. Let $b \in \varphi^{-1}(x)$. Then, $\varphi(b^n) \in \varphi(\ip) \imp b \in \ip$ (by observation 2). If $\mathrm{deg}(x) = 0$, then $\exists y \notin \varphi(\ip)$, $y \in T_d$, $d \geq d_0$ such that $xy \in \varphi(\ip)$. As $\varphi_d$ is an isomorphism, then $\exists! a \in S_d \cap \ip$ such that $\varphi(a) = xy$, and $\exists! b \in S_d, b \notin \ip$ such that $\varphi(b) = y$. Let $c \in \varphi^{-1}(x)$. Then $\varphi(cb) = \varphi(a) \imp cb = a \in \ip$ but $b \notin \ip \imp c \in \ip$. With this reasoning we have proved that $\varphi^{-1}(\iq) \subseteq \ip$. Reciprocally, given $x$ an homogeneous element of $\ip$, $\varphi(x) \in \varphi(\ip) \imp \varphi(x) \in \iq$ and this means that $x \in \varphi^{-1}(\varphi(x)) \in \varphi^{-1}(\iq)$. This proves that $\ip \subseteq \varphi^{-1}(\iq)$. In conclusion, we have found an homogeneous ideal $\iq$ such that $\varphi^{-1}(\iq) = \ip$, so $f$ is surjective.

			\item $f^{-1}$ is continuous: It is enough to check that the antiimage of a basic open subset is open. Let $h \in T$ and consider the basic open subset $D_+(h)$. Note that $D_+(h) = D_+(h^n)$, so if $h$ has degree $\neq 0$, for a given $n$, $h^n$ has enough degree such that $h^n \notin \ip \iff \varphi^{-1}(h^n) \notin \varphi^{-1}(\ip)$, and therefore, $(f^{-1})^{-1}(D_+(h)) = D(\varphi^{-1}(h^n))$. If $\mathrm{deg}(h) = 0$, then $h \in \ip \iff \exists y \notin \ip$ such that $yh \in \ip$. Then we can choose an element $y$ of degree atlesat $d_0$ by Observation 1, and therefore $(f^{-1})^{-1}(V(h)) = (f^{-1})^{-1}(V(yh)) = V(\varphi^{-1}(yh))$. Then, $(f^{-1})^{-1}(D_+(h)) = D_+(\varphi(yh))$. In conclusion, the antiimage of an open set is open, and $f^{-1}$ is continuous.
			Since now we have proven that $f$ is an homeomorphism. 

			\item $f^{\#}$ is an isomorphism of sheaves: First let's prove injectivity. Suppose that we have two applications $s,s' \in \locO_{\Proj(S)}(V)$ such that $f^{\#}(V)(s) = f^{\#}(V)(s')$. Now let $\ip$ be a prime of $f^{-1}(V)$ and $f(\ip) \in V$. Let $s(f(\ip)) = \frac{a}{t}$ and $s'(f(\ip)) = \frac{a'}{t'}$. Then, $(f^{\#}(V)(s))(\ip) =  (f^{\#}(V)(s'))(\ip) \imp \frac{\varphi(a)}{\varphi(t)} = \frac{\varphi(a')}{\varphi(t')}$. That happens if and only if $\exists y \notin \ip$ such that $y(\varphi(at')-\varphi(a't)) = 0$. Note that we can multiply this expression by elements of higher degree ans it is still zero. Then, by observation 1, we can suppose that $\mathrm{deg}(y) = d \geq d_0$ and therefore $\exists x \in S_d, \, x \notin f(\ip)$ such that $\varphi(at'x)-\varphi(a'tx) = 0$. As $a'tx$ and $at'x$ are homogeneous elements with degrees higher that $d_0$, then $\varphi$ is injective and $at'x = a'tx \imp x(at'-a't) = 0 \imp \frac{a}{t} = \frac{a'}{t'}$ in $S_(f(\ip))$. As this is valid for every prime $\ip$, $s = s'$, and $f^{\#}$ is injective. 

			To prove surjectivity we will prove surjectivity of the induced applications on stalks, that are $\varphi_{\ip}$. Indeed, given $\frac{a}{t} \in T_{(\ip)}$, $\exists y \notin \ip$ with $\mathrm{deg}(y) = d \geq d_0$. Then, $\frac{a}{t} = \frac{ay}{ty}$. As $\varphi_d$ is surjective $\forall d \geq d_0$ and $ay, ty \in T_d$, with $d \geq d_0$, then $\exists a', t' \in S_d$, $t' \notin f(\ip)$, such that $ay = \varphi(a')$ and $ty = \varphi(t')$. Then $\frac{a}{t} = \varphi_{\ip}(\frac{a'}{t'})$ and so $\varphi_{\ip}$ is surjective.
		\end{itemize}

		\item

		
	\end{enumerate}
\end{sol}