\documentclass[12pt]{article}

\usepackage{amsmath}
\usepackage{amsthm}
\usepackage{amssymb}
\usepackage{enumitem}
\usepackage{empheq}

\newcommand{\N}{\ensuremath{\mathbb{N}}}
\newcommand{\Z}{\ensuremath{\mathbb{Z}}}
\newcommand{\R}{\ensuremath{\mathbb{R}}}
\newcommand{\Q}{\ensuremath{\mathbb{Q}}}
\newcommand{\C}{\ensuremath{\mathbb{C}}}
\newcommand{\imp}{\ensuremath{\Rightarrow}}
\newcommand{\ip}{\ensuremath{\mathfrak{p}}}
\newcommand{\iq}{\ensuremath{\mathfrak{q}}}
\newcommand{\im}{\ensuremath{\mathfrak{m}}}
\newcommand{\ia}{\ensuremath{\mathfrak{a}}}
\newcommand{\ib}{\ensuremath{\mathfrak{b}}}
\newcommand{\Spec}{\ensuremath{\text{Spec}}}
\newcommand{\ima}{\ensuremath{\text{im}}}


\newcommand*\closure[1]{\overline{#1}}

\newtheorem{ex}{Exercise}[section]
\theoremstyle{definition}
\newtheorem*{sol}{Solution}

\DeclareFontFamily{U}{BOONDOX-calo}{\skewchar\font=45 }
\DeclareFontShape{U}{BOONDOX-calo}{m}{n}{
  <-> s*[1.00] BOONDOX-r-calo}{}
\DeclareFontShape{U}{BOONDOX-calo}{b}{n}{
  <-> s*[1.00] BOONDOX-b-calo}{}
\DeclareMathAlphabet{\mathcalboondox}{U}{BOONDOX-calo}{m}{n}
\SetMathAlphabet{\mathcalboondox}{bold}{U}{BOONDOX-calo}{b}{n}
\DeclareMathAlphabet{\mathbcalboondox}{U}{BOONDOX-calo}{b}{n}

\newcommand{\sF}{\ensuremath{\mathcalboondox{F}}}
\newcommand{\sG}{\ensuremath{\mathcalboondox{G}}}
\newcommand{\sA}{\ensuremath{\mathcalboondox{A}}}



\begin{document}

\title{Exercises Hartshorne}
\author{Oriol Velasco Falguera}
\maketitle

\section{Sheaves}

\begin{ex}
	Let $A$ be an abelian group, and define the constant presheaf associated to $A$ on the topological space $X$ to be the presheaf $U \mapsto A$ for all $U \neq \varnothing$, with restriction maps the identity. Show that the constant sheaf $\sA$ defined in the text is the sheaf associated to this presheaf.
\end{ex}

\begin{sol}
	Let $sF$ denote the constant presheaf. Let's first see that each stalk $\sF_P$ is a copy of $A$. Indeed, the elements of $\sF_P$ are represented by pairs $\langle U, s \rangle$, with $U$ open neighbourhood of $P$ and $s \in A$. As the restriction maps are the identity, two pairs $\langle U,s \rangle$ and $\langle V, t \rangle$ represent the same element if and only if $s = t$, so $\sF_P = A$.

	Let $s$ be an application from $U$ to $\bigcup_{P \in U} \sF_P$ satisfying properties (1) and (2) from the definition of associated sheaf. By (1), $s(P) \in \sF_P$ is an element of $A$, and therefore $s$ can be regarded as an application from $U$ to $A$ (that we will denote $s'$). In addition, let $B \subseteq A$. For each $P \in s'^{-1}(B), \, \exists V_{P}$ neighbourhood of $P$ such that $s'(V_P) = t \in B$. Then $s'^{-1}(B) = \bigcup_{P \in s'^{-1}(B)} V_P$ which is open. We have proved that the antiimage of every subset is open and therefore $s'$ is continuos with $A$ being given the discrete topology.

	Reciprocally, any countinuous application $s'$ from $U$ to $A$ can be regarded as an application $s$ from $U$ to $\bigcup_{P \in U} \sF_P$, defining $s(P) = s'(P) \in \sF_P$. This assignation guarantees that $s$ satisfies (1). In addition, for each $P \in U$, the set $V = s'^{-1}(s'(P))$ is an open neighbourhood of $P$ (by continuity of s'), and every $Q \in V$ has the same image $s'(P)$, which proves that $s$ satisfies (2).

	In conclusion, $\sF^{+}(U)$ is the group of continuous maps from $U$ into $A$, and therefore $\sF^{+}$ is indeed the sheaf $\sA$ defined in the text.
\end{sol}

\begin{ex}
	\begin{enumerate}[label=\alph*)]
		\item For any morphism of sheaves $\varphi: \sF \to \sG$ show that for each point $P$, $(\ker \varphi)_P = \ker(\varphi_P)$ and $(\ima \varphi)_P = \ima (\varphi_P)$.
		\item Show that $\varphi$ is injective (respectively surjective) if and only if the induced map on the stalks $\varphi_P$ is injective (respectilevy surjective) for all $P$.
		\item Show that a sequence $\dots \xrightarrow[]{} \sF^{i-1} \xrightarrow[]{\varphi^{i-1}} \sF^{i} \xrightarrow[]{\varphi^{i}} \sF^{i+1} \xrightarrow[]{} \dots$ of sheaves and morphisms is exact if and only if for each $P \in X$ the corresponding sequence of stalks is exact as a sequence of abelian groups.
	\end{enumerate}
\end{ex}

\begin{sol}
	\begin{enumerate}[label=\alph*)]
		\item $(\ker \varphi)_P = \{(U,s), \, s \in \ker(\varphi(U))\}$, modulo the usual germ equivalence relationship. Thus it can be regarded as a subset of $\sF_P$ as $\ker(\varphi(U)) \subseteq \sF(U)$. On the other side $\ker(\varphi_P)$ is a subset of $\sF_P$. To see that the two sets are equal it's enough to check the double inclusion. Let $\overline{(U,s)} \in (\ker \varphi)_P$. Then, $\varphi_P(\overline{(U,s)}) = \overline{(U, (\varphi(U))(s))} = \overline{(U,0)} = 0 \imp \overline{(U,s)} \in \ker(\varphi_P)$. Reciprocally, given $\overline{(U,s)} \in \ker(\varphi_P) \imp \exists V \subset U$ such that $\varphi(U)(s)|_V = 0$. As restrictions commute with morphisms of sheaves, $ \varphi(V)(s|_V) = \varphi(U)(s)|_V = 0$. Then, $\overline{(U,s)} = \overline{(V,s|_V)} \in (\ker\varphi)_P$. In conclusion, $(\ker \varphi)_P = \ker(\varphi_P)$.

		As $\sF_P = \sF^{+}_P$, $(\ima \varphi)_P$ is equal to the stack of the presheaf image at point P. $\ima (\varphi_P) = \{\overline{(U,s)} \in \sG_P \text{ such that } \exists \overline{(V,t)} \in \sF_P | \varphi_P(\overline{(V,t)} = \overline{(U,s)}\}$. But as $\varphi_P(\overline{(V,t)}) = \overline{(V, \varphi(V)(t))}$ then $\overline{(U,s)} \in \ima (\varphi_P) \iff \exists W$ neighbourhood of $P$, $W \subseteq V \cap U$ such that $\varphi(V)(t)|_W = s_W \iff \varphi(W)(t|_W) = s|_W \iff \overline{(U,s)} = \overline(W,\varphi(W)(t|_W)) \iff \overline{(U,s)} \in (\ima \varphi)_P$.

		\item $\varphi$ injective $\imp \ker \varphi = 0 \imp (\ker \phi)_P = 0 \, \forall P$. Using part a) of the problem, $\ker \varphi_P = (\ker \varphi)_P = 0 \imp \varphi_P$ is injective $\forall P$. Reciprocally, let $x \in \ker \varphi(U)$. $\forall P \in U, (\ker \phi)_P = 0$ so the image of $x$ in the stalk $(\ker \phi)_P$ is zero, which means that $\exists W_P \subseteq U$ neighbourhood of $P$ such that $x|_{W_P} = 0$. But open sets $W_P$ cover $U$ and therefore, by property (3) of the definition of shieves, $x = 0$. In conclusion, $\ker \varphi(U) = 0 \, \, \forall U \imp \ker \varphi = 0 \imp \varphi$ injective.

		We proceed similarly with the surjectivity. $\ima \varphi = \sG \imp (\ima \varphi)_P = \sG_P \imp \ima (\varphi_P) = \sG_P \imp \varphi_P$ surjective. To prove the other implication, First we will prove a fact that is stated but not proved in the text: $\sF^{+} \cong \sF$ if $\sF$ is already a sheaf. Given an open set $U$, let $V_P$ be the neighbourhood of $P$ contained in $U$ such that $\exists t \in \sF(V_P)$ such that $t_Q = s(Q) \, \forall Q \in V_P$. The sets $V_P$ cover $U$, and given two of these sets, $V,V'$ and the respective elements $t,t'$ we have that $\overline{(V',t')} = \overline{(V,t)}$ in every stalk $\sF_Q \imp \exists W_Q$ such that $t'|_{W_Q} = t|_{W_Q} \forall Q \in V \cap V'$. Then these $W_Q$ cover $V \cap V'$, and by property (3) applied to $t'|_{V\cap V'}-t_{V\cap V'}$, we have that $t_{V\cap V'} = t'_{V \cap V'}$. Then, by property (4) applied to the sets $V_P$, $\exists t \in \sF(U)$ such that $t_Q = s(Q) \, \, \forall Q \in U$, which means that each application $s$ is uniquely determined by $t \in \sF(U)$, and then $\sF^{+}(U) \cong \sF(U)$. Now it's easy to check that $\varphi$ is surjective. We have that $(\ima \varphi)_P = \ima (\varphi_P) = \sG_P$ and so we have that $\im \varphi (U)$ is the set of functions $s$ from $U$ to $\bigcup_{P \in U} \sG_P$, which means that $\ima \varphi$ is in fact $\sG^{+} \cong \sG$ as $\sG$ is already a sheaf.

		\item Given a sequence of sheaves and morphisms $\dots \xrightarrow[]{} \sF^{i-1} \xrightarrow[]{\varphi^{i-1}} \sF^{i} \xrightarrow[]{\varphi^{i}} \sF^{i+1} \xrightarrow[]{} \dots$, it is exact $\iff \ker \varphi^{i} = \ima \varphi^{i-1}$. If the sequence is exact, taking direct limits at both sides and using section a) we have that $\ker (\varphi^{i}_P) = (\ker \varphi^{i})_P = (\ima \varphi^{i-1})_P = \ima (\varphi^{i-1}_P) $, and so the sequence of stalks at each point $P$ is exact.

		The other implicaton is consequence of a more general result that we will prove now: Two subsheaves are equal $\iff$ the corresponding stalks at each point are equal. Let $\sF_1, \sF_2$ be two subsheaves of $\sF$, such that $(\sF_1)_P = (\sF_2)_P$. Let $t \in \sF_1(U)$. For every $P \in U \exists V_P$ neighbourhood of $P$ and $s \in \sF_2(V)$ such that $s_P = t_P$. The sets $V_P \cap U$ cover $U$, and given two of these sets, $V,V'$ and the respective elements $s,s'$ we have that $\overline{(V',s')} = \overline{(V,s)} = \overline{(U,t)}$ in every stalk $\sF_Q \imp \exists W_Q$ such that $s'|_{W_Q} = s|_{W_Q} \forall Q \in V \cap V'$. Then these $W_Q$ cover $V \cap V'$, and by property (3) applied to $s'|_{V\cap V'}-s|_{V\cap V'}$, we have that $s|_{V\cap V'} = s'|_{V \cap V'}$. Then, by property (4) applied to the sets $V_P \cap U$, $\exists r \in \sF_2(U)$ such that $t_P = r_P \forall P$. By property (3) applied to $r-t$ we get $s = t$ and so $t \in \sF_2(U)$. So $\sF_1(U) \subseteq \sF_2(U)$ and by symmetry $\sF_1(U) = \sF_2(U)$.
	\end{enumerate}
\end{sol}

\begin{ex}
	\begin{enumerate}[label=\alph*)]
		\item Let $\varphi: \sF \to \sG$ be a morphism of sheaves on $X$. Show that $\varphi$ is surjective if and only if the following condition holds: for every open set $U \subseteq X$, and for every $s \in \sG(U)$, there is a covering $\{U_i\}$ of $U$, and there are elements $t_i \in \sF(U_i)$, such that $\varphi(t_i) = s|_{U_i}$, for all i.

		\item Give an example of a surjective morphism of sheaves $\varphi: \sF \to \sG$ and an open set $U$ such that $\varphi(U): \sF(U) \to \sG(U)$ is not injective.
	\end{enumerate}
\end{ex}

\begin{sol}
	\begin{enumerate}[label=\alph*)]
		\item From problem 1.2, $\varphi$ is surjective $\iff$ the induced morphism on every stalk is. Suppose $\varphi_P$ is surjective, then given $U \subseteq X$ open set, $s \in \sG(U)$, $\forall P \in U \exists V$ neighbourhood of $P$ and $t \in \sF(V)$ such that $\overline{(U,s)} = \overline{(V,\varphi(t))} \imp \exists W_P \subseteq U \cap V$ such that $\varphi(t|_{W_P}) = s|_{W_P}$, and so $\{W_P\}$ is the covering that satisfies the desired property. Reciprocally, let $\overline{(U,s)} \in \sG_P$. Then $\forall P \in U \exists i$ such that $P \in U_i \imp \overline{(U,s)} = \overline{(U_i, \varphi(U_i)(t_i))} = \varphi_P(\overline{(U_i, t_i)}) \imp \varphi_P$ is surjective.

		\item
	\end{enumerate}
\end{sol}

\begin{ex}
	\begin{enumerate}[label=\alph*)]
		\item Let $\varphi: \sF \to \sG$ be a morphism of presheaves such that $\varphi(U): \sF(U) \to \sG(U)$ is injective foreach $U$. Show that the induced map $\varphi^{+}: \sF^{+} \to \sG^{+}$ of associated sheaves is injective.

		\item Use part (a) to show that if $\varphi: \sF \to \sG$ is a morphism of sheaves, then $\ima \varphi$ can be naturally identified with a subsheaf of $\sG$, as mentioned in the text.
	\end{enumerate}
\end{ex}

\begin{sol}
	\begin{enumerate}[label=\alph*)]
		\item Using 1.2 b) and the fact that $\sF^{+}_P = \sF_P$, the map $\varphi^{+}$ is injective $\iff$ the maps $\varphi_P: \sF_P \to \sG_P$ on the stalks are injective. Let $\overline{(U,s)} \in \sF_P$ such that $\varphi_P(\overline{(U,s)}) = 0 \imp \exists W \subset U$ such that $\varphi(U)(s)|_W = 0 \imp \varphi(W)(s|_W) = 0$. But as $\varphi(U)$ is injective $\forall U$, then $s|_W = 0$ and therefore $\overline{(U,s)} = \overline{(W,s|_W)} = 0$ and thus $\varphi_P$ is injective $\imp \varphi^{+}$ is injective.

		\item Let's consider the presheaf image of a morphism of sheaves $\varphi: \sF \to \sG$, $U \mapsto \ima (\varphi(U))$. Then for each $U$, $\ima (\varphi(U)) \subseteq \sG(U)$, and so the inclusion $i(U): \ima (\varphi(U)) \to \sG(U)$ is an injective morphism of abelian groups $\forall U$. Then, by section a), the induced map $i^{+}: \ima \varphi \to \sG^{+} = \sG$ is injective. 
	\end{enumerate}
\end{sol}

\begin{ex}
	Show that a morphism of sheaves is an isomphism if and only if it is both injective and surjective.
\end{ex}

\begin{sol}

\end{sol}

\begin{ex}
	\begin{enumerate}[label=\alph*)]
		\item Let $\sF'$ be a subsheaf of a sheaf $\sF$. Show that the natural map of $\sF$ to the quotient sheaf $\sF / \sF'$ is surjective, and has kernel $\sF'$. Thus there is an exact sequence
		\[
			0 \to \sF' \to \sF \to \sF / \sF' \to 0
		\]

		\item Conversely, if $0 \to \sF' \to \sF \to \sF / \sF' \to 0$ is an exact sequence, show that $\sF'$ is isomorphic to a subsheaf of $\sF$ and that $\sF''$ is isomorphic to the quotient of $\sF$ by this subsheaf.
	\end{enumerate}
\end{ex}

\end{document}